\documentclass{article}
\usepackage[english]{babel}
\usepackage[utf8]{inputenc}
\usepackage[T1]{fontenc}
\usepackage{hyperref}
\usepackage[margin=0.5in]{geometry}
\usepackage{xspace}
%\usepackage{siunitx}
\usepackage{xcolor}
\usepackage{amsmath}
\usepackage{graphicx}
\usepackage{palatino}
\usepackage{microtype}

\newcommand{\sir}{\textsc{sir}\xspace}
\newcommand{\gspn}{\textsc{gspn}\xspace}

\title{SIR}
\author{Drew}

\begin{document}
\maketitle

\section{Introduction}
This document is about a little \sir simulation along the lines of the
time-dependent infection rate example from Wearing and others\cite{Wearing2012}.
We'll use the Semi-Markov library\cite{SemiMarkov2014} to do the simulation.

\section{ODE Model}
The model, as given in Wearing, is an \sir with birth and death.
Using $N=S+I+R$, $\mu$ as a death rate, $\gamma$ as a recovery rate,
and $\beta I/N$ as a freqency-dependent infection rate,
\begin{eqnarray}
  \frac{dS}{dt}&=& B - \frac{\beta I}{N}S-\mu S \\
  \frac{dI}{dt}&=& \frac{\beta I}{N}S-\gamma I-\mu I \\
  \frac{dR}{dt}&=& \gamma I - \mu R.
\end{eqnarray}
The crude birth rate, $B$, is an inhomgeneous term which fixes
asymptotic population size by breaking
scaling invariance, which you can see by adding the equations,
\begin{equation}
  \frac{dN}{dt}=B-\mu N.
\end{equation}
At long times and for fixed $\beta$,
\begin{eqnarray}
  N & = & B/\mu \label{eqn::longtime0}\\
  S & = & \frac{(\mu+\gamma)B}{\beta \mu} \\
  I & = & \frac{(\beta-\mu-\gamma)B}{\beta(\mu+\gamma)} \\
  R & = & \gamma I/\mu\label{eqn:longtime}
\end{eqnarray}
The model seems well-behaved. We are interested in adding a seasonal
dependency to this model, so $\beta=\beta_0(1+\beta_1 \cos(2\pi (t-t_i))).$
Time is measured in years, here. The values used in the example
are \\
\begin{tabular}{ll}
B & 1/70 per year \\
$\beta$ & 400 per year \\
$\mu$ & 1/70 per year \\
$\gamma$ & 365/14 per year.
\end{tabular} \\
Note that $N=1$ with the choice of $B/\mu$. Given these values, the
Eqs.~\ref{eqn::longtime0}--\ref{eqn:longtime} predict
$S=0.07$, $I=0.0005$, $R=.93$, which looks like what is in the graph
called ``$R(0)=15.33$.''

\section{Stochastic Model}
\subsection{Model Definition}
For background on stochastic \sir, see Allen's chapter\cite{Allen:2008ma}.
Every transition in the stochastic model is exponential, which makes for
a simple system. There are three places, $S$, $I$, and $R$, with a transition
for each term of the differential equation. The \gspn is shown in Fig.~\ref{fig:sirgspn}.
\begin{figure}
\centerline{\includegraphics[scale=0.35]{sirexpgspn}}
\caption{The \gspn for the stochastic model. Note that the infection
transition includes inputs from all of $S$, $I$, and $R$ because it depends
on the total $N$ individuals.\label{fig:sirgspn}}
\end{figure}
Working from the figure, we list our transitions. \\
\begin{tabular}{ll}
  \hline
  Transition & Hazard Rate \\
  \hline
  birth & $B$ \\
  death-s & $\mu S$ \\
  death-i & $\mu I$ \\
  death-r & $\mu R$ \\
  infect & $\beta SI/(S+I+R)$ \\
  recover & $\gamma I$ \\
  waning immunity & $w R$ \\
  \hline
\end{tabular}


\subsection{Distribution for Seasonal Hazard Rate}
The only complication is that the hazard rate will be seasonal,
\begin{equation}
  \beta(t)=\beta_0(1+\beta_1 \cos(2\pi (t-t_i))),
\end{equation}
where $t_i$ is an offset into the year and $\beta_1<1$. It is likely there will be enough
events per year that we can safely approximate this piecewise, but let's start
with the exact distribution because we can.
We can write this as
\begin{equation}
  \beta(t)=\beta_0(1+\beta_1 [\cos(2\pi t) \cos(2\pi t_i)+\sin(2\pi t) \sin(2\pi t_i)]).
\end{equation}
This hazard depends on the absolute system time but does not depend on
when an individual entered the susceptible state. We construct the
distribution for this transition, from the hazard rate, as
\begin{equation}
  F(t,t_0)=1-e^{-\int_{t_0}^t\beta_0(1+\beta_1 \cos(2\pi (s-t_i)))ds},
\end{equation}
which we will need to invert, $U=F(t,t_0)$, in order to sample.
Looking just at the integral,
\begin{equation}
  \int_{t_0}^t\beta_0(1+\beta_1 \cos(2\pi (s-t_i)))=\beta_0\left((t-t_0)+\frac{\beta_1}{2\pi}\sin(2\pi (t-t_i))-\frac{\beta_1}{2\pi}\sin(2\pi (t_0-t_i))\right)
\end{equation}
If we define $t_0'=t_0+\frac{\beta_1}{2\pi}\sin(2\pi (t_0-t_i)),$ then we
have to solve
\begin{equation}
  -\ln(1-U)/\beta_0+t_0'=t+\frac{\beta_1}{2\pi}\sin(2\pi (t-t_i)).
\end{equation}

% s(a-b)=sa cb - ca sb
% s(t-i)-s(o-i)=st ci - ct si -(so ci- co si)
% st ci - ct si - so ci + co si
% (st-so) ci - (ct-co) si

% s(t-o)=st co - ct so
% c(t-o)=ct co - st so

% mult by co
% st co ci - ct co si - so co ci + co co si

% add 

\section{Sample Runs and Timing}
Using the initial values from the Eq.~\ref{eqn:longtime}, the graph
in Fig.~\ref{fig:longbehavior} shows a long time equilibrium value
which is the same as that for the \textsc{ode}.
\begin{figure}
\centerline{\includegraphics[scale=1.0]{fixed_no_seasonal}}
\caption{Initial values from \textsc{ode} are the same as the long-time
equilibrium of this stochastic model of 100,000 individuals.
Here there is no seasonality.\label{fig:longbehavior}}
\end{figure}

\begin{figure}
\centerline{\includegraphics[scale=1.0]{feeling_better}}
\caption{Another trajectory, this time for no birth or death.
It looks as we hope for SIR.\label{fig:longexact}}
\end{figure}

\begin{figure}
\centerline{\includegraphics[scale=0.8]{tenmillion}}
\caption{When we go to 10,000,000 individuals, the results look like the
ODE.}
\end{figure}


\begin{figure}
\centerline{\includegraphics[scale=0.8]{smoothedtenmillion}}
\caption{A closeup of the smoothed trajectory for 10,000,000 individuals
shows how seasonality affects the number of infected.}
\end{figure}

Runtime for the model, covering $30$~years, is shown in the table below.
The results come from \texttt{python runner.py --time --runcnt 10}. \\
\begin{tabular}{cc}
\# individuals & wall clock seconds \\
5000 & 0.013 \\
10,000 & 0.025 \\
20,000 & 0.050 \\
50,000 & 0.14 \\
100,000 & 0.37 \\
200,000 & 1.0 \\
500,000 & 5.3 \\
1,000,000 & 19
\end{tabular} \\
These values assume a single thread, but the code is multi-threaded
using the \texttt{-j} flag. For instance, running 100, 30-year trajectories
of 200,000 individuals on my 4-core machine takes about 30 seconds.

\section{Running the Executable}
Command-line options include all of the parameters to the system, plus the following:
\newcommand{\opt}[1]{\texttt{#1}}
\begin{description}
  \item[\opt{-j}] The number of threads to use.
  \item[\opt{-{}-runcnt}] How many times to run the simulation.
  \item[\opt{-s}] The total number of individuals. Not the number of susceptibles,
   surprisingly. Want that changed?
  \item[\opt{-i}] Total number of infecteds.
  \item[\opt{-r}] Total number of recovereds.
  \item[\opt{-{}-seed}] Starting random seed. Multithreaded runs use a seed per thread, so don't just increment a single seed.
  \item[\opt{-{}-endtime}] How many years to run.
  \item[\opt{-{}-exacttraj}] There are two observers of the trajectory. One saves
  every step. The inexact one only saves steps when they change $.1\%$ of the value.
  \item[\opt{-{}-exactinfect}] Whether to use the time-dependent infection rate
  or the inexact version, which uses a piecewise approximation. This exists so that
  we can put an upper bound on the error to the obviously good choice of a piecewise infection rate.
  \item[\opt{-{}-datafile}] Filename to which to save. This is overwritten each time.
  We can change that, too.
  \item[\opt{-{}-loglevel}] Whether to print logging information. You can quiet the program significantly with \opt{-{}-loglevel=error}.
  \item[\opt{-{}-info}] Show how the program was built, meaning its git repository
  and makefile.
\end{description}

Files in the directory:
\begin{description}
  \item[default\_parser.py] Puts some default arguments into Python's argparse module.
  \item[freqsir.*] This document.
  \item[getgit.py] Retrieves the git repository information from the current
  directory and places it into a C++ header file so the program can know
  where it came from.
  \item[hdf\_file.\{hpp,cpp\}] The file format uses HDF5. This class
  uses a C++ PIMPL idiom. Note the mutex which guarantees only one
  writer at a time, so multiple threads can hold this object.
  \item[main.cpp] The C++ main().
  \item[Makefile] The SMVHIDELOG option removes all tracing code, which
  is about ten percent or more of the run time. See this file for the
  list of required libraries.
  \item[quick.py] Plotting trajectories using Matplotlib.
  \item[runner.py] Runs timing. Also a demonstration of how to read
  the HDF5 file.
  \item[seasonal.\{cpp,hpp\}] Implements a statistical distribution
  whose hazard rate is the seasonal hazard for infection. Uses
  GSL's root-finding methods to solve for the sample.
  \item[sir\_exp.\{cpp,hpp\}] This is the GSPN model! Everything
  scientific is in this file, between lines 100 and 400.
\end{description}

\bibliography{freqsir}
\bibliographystyle{plain}
\end{document}


